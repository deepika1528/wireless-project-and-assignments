\documentclass[12pt]{article}

% --------------------
% Basic Packages
% --------------------
\usepackage[a4paper,margin=1in]{geometry}
\usepackage{setspace}
\usepackage{graphicx}
\usepackage{amsmath}
\usepackage{amssymb}

\onehalfspacing

% --------------------
% Title and Author
% --------------------

\title{Low-Latency MAC Protocols for Real-Time
Wireless Control Systems}

\author{Mani Deepika Arja}

\date{}

\begin{document}
	\maketitle

%name: Mani deepika Arja
%reg no: 23MIC7296
%gamil: manideepika4114@gamil.com 
%vitap mail: deepika.23mic7296@vitapstudent.ac.in
%orcid:0009-0005-0355-6834

    
	\section{Introduction}

\subsection{Motivation for Real-Time Wireless Control}

Real time wireless control systems are being used in lots of automation places now. The delay in communication is really important in these systems because it has an effect on how the machines work. You see sensors send information to the controllers. Then the controllers send commands back to the actuators. If the information gets delayed for long, the control process for the wireless control systems may not work like it should. Wireless control systems need everything to happen quickly so they can work properly.

Industrial control networks are different from networks. They cannot just send information and Hope it gets there. Industrial control networks need to know when the information will arrive. This is because industrial control networks need timing. Some recent studies on networking for industrial control networks show that having a structured way of accessing the network helps in keeping the timing stable \cite{Rico-Menendez202554}. Because of this, researchers are working hard on designing protocols, for industrial control networks. These protocols are called MAC protocols. The goal of these MAC protocols is to reduce delay and make the network more consistent. Industrial control networks need these MAC protocols to work well.

People like to use wireless communication in a lot of situations. This is because it gets rid of a lot of wires and makes it easier to design systems. Wireless networks can also cause problems like interference and collisions. Sometimes they even have to resend information again. All these things can make delays happen at many times.

When we talk about industrial IoT scheduling we see that it is really important to keep delays the same. This is often more important, than trying to send a lot of data at once \cite{vanLeemput2024180034}. So when we make MAC protocols for real-time systems we need to focus on making sure things happen in an order. We cannot just try to send as much data as possible.

Industrial systems these days need to be flexible. Production lines are often changed. Wireless communication helps with this flexibility. At the same time it is hard to keep the timing of things happening in a predictable way when the traffic conditions are changing.

Research on industrial communication systems shows that the way we access the system and the control requirements have to work closely together to make sure we have flexibility and that things happen on time \cite{Zeydan20256808}. This is why designing a system that can send information quickly called a low-latency MAC is an important area of research, for industrial systems.

\subsection{Role of the MAC Layer in Latency Control}

The MAC layer is really important because it decides how different devices use the channel. So when a lot of devices try to send information at the same time they can interfere with each other. This is called a collision. When collisions happen it takes longer for devices to send and receive information. It also makes it hard to predict when things will happen.

In IoT networks this can be a big problem because it can affect how well the system is controlled. Some people did research on something called slot allocation in 6TiSCH systems. They found out that when devices take turns sending information in a structured way it reduces delays and makes timing more consistent \cite{Kalita20249365}. This just shows that the MAC layer is very important for controlling latency.

Traditional contention-based protocols are good for communication but they do not promise that messages will be delivered on time.For systems that need to work with timing people usually use slot-based or scheduled access.

The Wi-Fi-based time-sensitive networking studies show that better scheduling can really help, even when things are moving around \cite{Avila-Campos202430687}.This shows how important it is to have a MAC design for wireless systems used in industries.

The MAC layer also affects how reliable something is. It has rules for sending things handling priority and making reservations and all these things can make a difference in how long things take. In networks where things have to go through many hops even small delays can add up. People who study how to make communication predictable in wireless systems say that when we make decisions about access we have to think about what kind of traffic it is and when it needs to be done \cite{Sharma2023106898}. So modern MAC protocols use scheduling that is structured and coordination that can adapt to keep everything stable.

\subsection{Impact of Latency on Control Stability}

Communication delay has an impact on how well a system is controlled. When sensored data is late the controller makes decisions based on outdated information. This can lead to problems over time like reduced accuracy or even instability.

For wireless systems it is really important to have strict rules about when data arrives so the system can stay stable \cite{Sharma2023106898}.

Communication delay directly influences control performance of the communication system. Stable timing is therefore essential, for the communication system and sensor data.

Jitter is one of the problem that mainly occurs here. Jitter means that the time it takes for packets to arrive can be different each time. This is also very important. Many control algorithms need to get updates at the same time every time. If the timing is not the same every time the system may not work well as it should. People have done research on timing services in wireless systems and they found out that when everything is synchronized properly it helps to reduce the variation in delay \cite{Chandramouli202335150}. Jitter can be reduced with MAC protocols that do not make things wait for unpredictable amounts of time.

In systems where lots of devices work together delays and timing mistakes can add up. Even if the average delay seems okay, strange timings can affect processes. Some research on deterministic networking says that controlling variations in delay or jitter is sometimes more important than just reducing the average delay \cite{Rico-Menendez202554}. So the rules that control how devices communicate called MAC protocols must make sure that messages are sent at predictable times especially for real-time control systems.

\subsection{Research Scope and Objectives}

This paper looks at low latency MAC protocols for wireless control systems that need to work in time. The paper is focused on deterministic scheduling controlling contention in an adaptive way and using hybrid methods to get access. Research on 6TiSCH management found that when things are coordinated in a structured way it helps to make delays more stable in industrial settings \cite{Graf2026382}. These mechanisms are important for wireless control systems to have communication that is predictable.

Apart from designing protocols this study also thinks about scalability and integration with wireless systems. Looking at research on scheduling over time-sensitive networks it seems that giving resources in a smart way makes communication more consistent \cite{Yang20245162}. This paper looks at what other people have written in journals to see how the design of the medium access control layer helps make communication reliable and fast.

Industrial systems must also integrate with emerging technologies such as 5G and time-sensitive networking. Survey research on convergence between these technologies shows that cross-domain coordination is becoming important \cite{Sasiain2025259}. Therefore, this study not only examines performance aspects but also looks at practical deployment considerations.


	\section{Background and Conceptual Foundations}

\subsection{Architecture of Real-Time Wireless Control Systems}

At times, wireless control system is made up of lots of things like sensors and controllers and actuators all connected to each other through a wireless network. The sensors do the job of taking measurements from the world around us and they send this information to the controllers. Then the controllers figure out what to do. Send commands to the actuators. They have to do all this within a time limit.

In factories and industries these days how things communicate with each other is not separate, from the control system. It is a part of the control system itself. The time wireless control system and its communication behavior are connected. Research on the convergence of 5G and time-sensitive networking technologies highlights the importance of aligning communication scheduling with control loop requirements \cite{Sasiain2025259}.

When we actually use networks they are usually connected to wired parts to make sure everything happens on time. The 5G system works like a bridge between devices and special networks that need to be very precise with their timing. This means that data can be sent back and forth between types of connections in a synchronized way.

The 5G system and wireless networks can work together with parts so industrial controllers can talk to both wired and wireless parts easily and still meet the strict rules, about how long things can take. So when we think about wireless and TSN infrastructures we need to consider how they work together. This is really important for making sure that automation systems can communicate in time. The integration of wireless and TSN infrastructures is the basis for getting real-time communication, in modern automation systems.

The system is made up of a lot of parts. How they all work together is shown in Figure~\ref{fig:architecture}. One thing that really matters is the communication delay because it has an impact, on the feedback loop of the system.

Industrial deployments often have to deal with tough radio conditions. These conditions include interference, multipath fading and signal obstruction. This means they have retransmissions and delay variability.

Some studies on ultra-wideband real-time industrial connectivity were done \cite{Kabaci2025187976}. These studies show that ultra-wideband real-time industrial connectivity needs strong MAC-layer coordination. This coordination is necessary for short-range wireless solutions. It helps maintain performance in industrial deployments.

The delay in the network and the performance of the control system are connected. So we need to consider the whole system when designing industrial deployments. This is called cross-layer design awareness, for ultra-wideband real-time industrial connectivity and industrial deployments.

\begin{figure}[htbp]
\centering
\includegraphics[width=0.85\textwidth]{architecture.png}
\caption{Architecture of a real-time wireless control system including sensors, controller, wireless network and actuators. Adapted based on concepts discussed in \cite{Sasiain2025259}.}
\label{fig:architecture}
\end{figure}

\subsection{Latency Components in Wireless Networks}

The delay in wireless control systems is made up of different parts: the time it takes to send something, the time it takes to process it, the time it waits in line and the time it takes to get access to the medium. Out of all these parts the time it takes to get access to the medium is usually the hardest to predict. This is because it depends on how all the nodes compete with each other to get access to the channel.

Some research on using reinforcement learning to schedule uplink traffic in networks that are sensitive to latency shows that making smart decisions about scheduling can reduce how much the access delay varies \cite{Robaglia20241142}. Managing how nodes access the medium is really important if we want to make sure that the communication works in a predictable way.

In networks with multiple hops or in systems that are spread out like industrial networks the time it takes for things to go through each hop can make the whole process even slower. Some people have figured out ways to make sure resources are used in a way that helps with timing in systems that need to be fast like those used in cars. For example resource reservation strategies have been shown to reduce deadline violations in time-sensitive vehicular networks \cite{Al-Khatib2024139649}. This shows that we need to be careful when we make the rules for how devices talk to each other in systems where timing is very important.

The delay in getting access to a network is affected by how many devices are on the network and the rules for deciding which devices get to send data first. Research on autonomous slot allocation in industrial IoT networks shows that scheduling in a structured way reduces how much the delay can vary compared to contention-based access \cite{Kalita20249365}. Medium access delay is something that we need to think about carefully.

\subsection{Fundamental MAC Design Principles}

When we design protocols for wireless control systems that need to work in time we have to make sure that the wireless channel is accessed in a predictable way. Devices must be able to send and receive data at defined times. We also need to make sure that the data gets through reliably.

Research on AI-assisted radio resource management in next-generation wireless networks shows that delay bounds must be maintained especially for critical communication \cite{Mumtaz202596198}. Deterministic access control helps reduce collisions on the network. By avoiding collisions we can make sure that devices have guaranteed opportunities to transmit data.

When we think about communications in industrial automation, energy efficiency and scalability are very important. Studies on industrial automation architectures highlight the need to balance latency performance with system complexity and scalability \cite{Zeydan20256808}. The MAC protocol design needs to consider latency reduction, synchronization overhead and network expansion capability.

Modern MAC protocols include mechanisms like guard intervals and synchronization correction techniques. Research on timing services in advanced wireless systems shows that precise clock alignment is important for scheduled communication to work properly \cite{Chandramouli202335150}. When these principles are applied in MAC-layer design it improves reliability in changing industrial environments.

\subsection{Performance Evaluation Metrics}

When we evaluate MAC protocols for wireless control systems we need to focus on timing consistency not just speed. End-to-end delay and jitter are very important for closed-loop control. Research on cloud-fog automation frameworks shows that predictable communication improves coordination in industrial environments \cite{Jin20252917}.

We also need to consider reliability and deadline compliance. Research on deterministic federated learning over time-sensitive networks shows that intelligent resource allocation helps reduce missed deadlines \cite{Yang20245162}. By combining latency, jitter, reliability and deadline metrics we can better understand MAC protocol performance.

The main metrics used to evaluate low-latency MAC protocols are summarized in Table~\ref{tab:metrics}.

\begin{table}[htbp]
\centering
\caption{Performance metrics used to evaluate low-latency MAC protocols in real-time wireless control systems, based on concepts discussed in \cite{Rico-Menendez202554}.}
\label{tab:metrics}
\renewcommand{\arraystretch}{1.4}
\begin{tabular}{|p{3.5cm}|p{5.5cm}|p{4cm}|}
\hline
Metric & Description & Relevance to Control Systems \\
\hline
End-to-End Delay & Time required for a data packet to travel from sensor to actuator & Determines responsiveness and stability of the control loop \\
\hline
Jitter & Variation in packet transmission delay across communication cycles & Affects timing predictability and synchronization accuracy \\
\hline
Packet Delivery Ratio & Percentage of successfully delivered packets within the communication interval & Indicates reliability of wireless transmission \\
\hline
Deadline Miss Ratio & Proportion of packets exceeding predefined timing constraints & Direct measure of control performance degradation \\
\hline
\end{tabular}
\end{table}



	\section{Core Concepts and Approaches}

\subsection{Deterministic Scheduling Mechanisms}

Deterministic scheduling is the base of low latency MAC protocol design for real time wireless control systems. This means that devices get a chance to send data at a time. We plan this in advance so every device knows when it is their turn. This way we can predict when a device can use the channel.

Some people made a system for 6TiSCH-based industrial IoT networks that automatically gives time slots to devices \cite{Kalita20249365}. This shows that scheduling can really reduce delays and make the system more stable. Deterministic scheduling is good because it stops devices from sending data at the time and reduces the need to resend data. This makes the timing of real time systems more stable. That is important, for critical applications.

Synchronization is really important for communication that always works the way. If the timing is off between devices it can cause problems like overlapping and big delays. People have studied how timing services are getting better for wireless systems and they think it is very important to have good synchronization systems for networking that always works the same way \cite{Chandramouli202335150}. When all the clocks are working together perfectly it makes sure that everything runs smoothly and works the way every time in wireless systems used in industries. Synchronization is key, to making sure that wireless systems work well.

Multi-channel diversity is something that deterministic MAC frameworks often have along with fixed slot allocation. This helps make them more robust.

When systems spread out scheduled transmissions across frequency channels they can deal with interference better and reduce the chance that they will have to transmit something again.

In sensor networks people have found that using channel hopping and structured slot allocation together can make things more reliable and stable when it comes to latency.For example  \cite{vanLeemput2024180034}.These kinds of mechanisms are really useful, in environments where there are a lot of machines close together and electromagnetic interference is a common problem.

Communication that is deterministic and based on slots is usually set up in a repeating pattern. This pattern is made up of something called slotframes. Each slot in the slotframe is like a time slot. It is a chance to send information. It always lasts for the same amount of time. The deterministic slot-based communication uses these slotframes to make sure everything runs smoothly. In the slot-frames, each slot represents a fixed-duration transmission opportunity, for the slot-based communication.

In internet of things environments, autonomous slot allocation mechanisms are really useful. They show that managing slotframes in a way makes a big difference in terms of delay stability and scalability \cite{Kalita20249365}. This kind of access is very important for maintaining consistent performance, in systems that are controlled from many different places.

Recent advancements in Wi-Fi 8 also show that more and more people are supporting the use of Wi-Fi 8 for important jobs that need reliable wireless communication \cite{singh2025wi}. This is the case for things like Wi-Fi 8 that require a connection as seen in the work of Singh in 2025, about Wi-Fi 8.

Guard intervals are put in between scheduled transmissions. This is done to make up for mistakes in synchronization and differences in how it takes for things to travel.

These guard intervals stop transmissions from overlapping with each other. This means that the transmissions do not interfere with one another.

Guard intervals do take away a bit of bandwidth. They make the timing of the transmissions more reliable. This is especially true in places where the clocks are not perfectly in sync and where there is interference.

People have done research, on networks that have many different parts. This research shows that using the right guard intervals helps keep delays under control even when the network is made up of different kinds of communication systems \cite{Rico-Menendez202554}. So the slot duration and the guard time need to be set up right. This is really important when we are talking about low-latency MAC systems. We have to get the slot duration and the guard time right or it will not work very well. The slot duration and the guard time are critical, in low-latency MAC systems.

So I want to show you an example of how scheduling works in 6TiSCH networks. We use something called slotframe-based scheduling in 6TiSCH networks. You can see what this looks like in Figure~\ref{fig:6tisch}. Basically we assign times when devices can send data within repeating time cycles in 6TiSCH networks. This way devices in 6TiSCH networks do not interfere with each other when they send data so we have collision-free communication, in 6TiSCH networks.

\begin{figure}[htbp]
\centering
\includegraphics[width=0.85\linewidth]{6tisch_slots.png}
\caption{Example of slot scheduling in a 6TiSCH network using slotframe-based scheduling concepts discussed in \cite{Kalita20249365}.}
\label{fig:6tisch}
\end{figure}

\subsection{Contention Optimization Strategies}

Deterministic protocols are good at keeping delays under control. Contention-based mechanisms are better at handling changing situations. The problem with contention-based mechanisms is that they can make things take longer than expected if they are not managed properly.

Some methods that use reinforcement learning to schedule uplink transmissions have shown that they can reduce delays in situations where reliability and speed are crucial \cite{Robaglia20241142}. For example these methods can adjust transmission parameters to reduce latency.

Adaptive contention window tuning is a way to make nodes work well with different amounts of traffic. This means that nodes can adjust to changing traffic loads and still keep delays at a level. Adaptive contention window tuning is useful, for contention-based mechanisms because it helps nodes respond to traffic loads in a way.

Hybrid resource reservation frameworks are really good at helping to control contention. They do this by organizing how devices access the channel, which is especially important for time-sensitive traffic. Some experiments with deterministic wireless multi-domain networking show that when you carefully plan how devices access the network it reduces delays and makes the network more reliable \cite{Rico-Menendez202554}. When you combine methods that adapt to changing conditions with structured access you can optimize how devices contend for access, to the network. This means that hybrid resource reservation frameworks can improve how well the network works without making it too hard to add devices to the network.

\subsection{Hybrid and Adaptive Access Models}

Hybrid MAC approaches are really useful because they bring together two ways of managing time slots. They use a fixed schedule for some things and a flexible approach for others. So in these systems things that need to happen at the time every time get their own special time slots. Then the time that is left over is used for things that need to happen when something specific occurs.

In wireless sensor networks people have found that using energy-aware adaptive scheduling strategies is a good idea \cite{vanLeemput2024180034}. This means that the system can adjust to what is happening in real time. For example if something needs to happen right away the system can make room for it. This approach is shown to work in a study by vanLeemput.

The good thing about this balance is that it lets us predict when things will happen which is really important in some systems. At the same time it makes sure that we are using our resources in the best way possible. Hybrid MAC approaches are all, about finding this balance and making the system work overall.

Adaptive resource management is really helpful in networking environments. It helps with scalability and integration with wireless architectures. Some research on scheduling in time-sensitive networks shows that managing resources in a flexible way makes communication more consistent in systems that are spread out \cite{Yang20245162}. So using hybrid and adaptive MAC models is a way to support different types of traffic in real-time wireless control networks. This is because these models can handle the varying needs of wireless control networks and support mixed traffic patterns. Adaptive resource management and hybrid MAC models are important, for networking environments and real-time wireless control networks.

\subsection{Time-Sensitive Networking Features in Wireless MAC}

Time sensitive networking principles were first made for wired Ethernet to be very predictable. Now these principles are also being used to design MAC protocols. This means that things like scheduling that considers time and controlling the flow of traffic are being used to make sure that wireless systems in industries have low latency. People are studying how 5G and Time Sensitive Networking technologies can work together. They want to see how wireless parts can work like bridges to make sure that behavior is predictable when it is not wired \cite{Sasiain2025259}. This makes it possible for different types of communication systems to work together in a synchronized way. Time sensitive networking principles are very important, for this.

One of the key mechanisms derived from TSN is the use of scheduled transmission windows that isolate critical control traffic from background data. By allocating dedicated time intervals for high-priority streams, MAC protocols can prevent interference from lower-priority transmissions. Studies on deterministic communication in advanced wireless frameworks indicate that structured scheduling combined with resource reservation significantly improves delay predictability \cite{Sharma2023106898}. Integrating TSN-inspired mechanisms into wireless MAC layers therefore strengthens the reliability of real-time control systems.


	\section{Comparative Discussion}

\subsection{Determinism versus Flexibility}

Deterministic MAC protocols are made to decide when each node can send information. They do this by setting up time slots of time so every device knows exactly when it is allowed to use the channel. This means that devices can access the channel in a structured way. Because of this it becomes easier to know when something will happen and the time between transmissions is more consistent. For things like control, where even tiny differences in timing can change how machines work deterministic MAC protocols are really helpful because they make it possible to predict what will happen. Deterministic MAC protocols are very good at reducing variation between transmissions, which's important, for industrial control applications. Research on systems that are coming next shows that making a plan, for when things are sent really helps these systems work better when time is very important. This is what some studies found out like the one done by Sharma.

Industrial networks are always changing. The traffic conditions can be different depending on what's being produced or how many devices are being used. So having a fixed plan for managing the network is not always the idea. This is because it can limit the networks ability to adapt to situations.

Recently people have been trying out approaches, especially with 5G and time-sensitive networking. These approaches use scheduled access for important traffic and more flexible methods, for less important data. The goal of these designs is to make sure the network can still work well even when unexpected things happen with the traffic. Industrial networks and time-sensitive networking need to be able to handle these changes.

When we look at how something performs, scheduling that always does the same thing usually gives us a better idea of the longest time we have to wait. Other methods that deal with competition for the network can work okay when not many people are using it. When a lot of people are using it at the same time it can take a long time to get something done and we cannot predict how long it will take.

People who study how to share resources in a way in systems that need to do things on time found out that planning when things can be sent helps make sure things get done on time even when the network is very busy.

So when we are looking at MAC protocols for real-time control we need to think about more, than how long things take on average we need to think about MAC protocols and how they affect real-time control and we need to consider MAC protocols. Worst-case delay behavior must also be examined to ensure stable operation in industrial environments.

\subsection{Latency Stability and Reliability Trade-offs}

Latency stability is really important in real-time wireless control systems. It is more important than the average delay. When we use scheduling it helps reduce the variation in delay. This happens because it minimizes collisions and retransmissions.

For example Cloud-fog automation frameworks show us how structured communication can improve coordination across industrial components \cite{Jin20252917}.

When the communication timing is stable it directly improves the accuracy of the control loop and the precision of the actuator in wireless control systems. This is because stable communication timing is crucial, for wireless control systems.

When we talk about communication we have to think about the speed at which things happen. Some ways of doing things can make sure that things happen at the speed on average but sometimes they can get really slow when a lot of things are happening at the same time. People who study how to make mobile communication work for factories and things like that say that we need to make sure that our systems are reliable which means they work all the time and that we also need to control how long things take to happen.

We can use a different methods to make our systems more reliable like sending the same information multiple times making sure important things get done first and saving resources just for the things that need them. These methods can help make sure that our information gets where it needs to go. They can also make our system more complicated and use more resources. So we need to find a balance between making sure things happen quickly and smoothly that they work all the time and that we do not waste resources.

To get good communication with almost no delays we often need to add some extra help and send information through many paths. We have found that setting aside resources just for very important traffic can really help meet deadlines \cite{Al-Khatib2024139649}. However if we set aside many resources it can limit the amount of bandwidth available for things that are not as important. So when we design the medium access control or MAC for short we need to make sure it is good at keeping the communication reliable while also using the spectrum in a smart way for all the different types of traffic in an industrial setting.

\subsection{Scalability and Deployment Considerations}

Scalability is really important when you have a lot of devices in industrial environments. When you do not have that devices, a simple way of giving each device a specific time slot works well.. When you have a lot of devices this method can be hard to manage.

Research on 6TiSCH network management shows that grouping devices and using a structured way of coordinating them can help scalability without slowing things down \cite{Graf2026382}. These management strategies that work across devices make the system more reliable in big industrial settings.

Hybrid scheduling approaches are really helpful when we have types of traffic all mixed together. For example research on networks that have many different parts working together shows that planning how resources are used across these parts can make things more reliable without making them too rigid \cite{Rico-Menendez202554}. When we are actually building these networks we have to think about how many devicesre connected how much time it takes to get everything working together and how important it is to have control over the network.

\subsection{Implementation Complexity and Resource Overhead}

The deterministic and hybrid MAC protocols are good because they help with delays.. When we try to use them in real life it gets complicated. We need to make sure everyone is on the page and that is hard to do. This is because we have to plan out when each device can send information. We also have to keep an eye on the network all the time.

People who study how to manage 6TiSCH deployments have found that we need to be careful about how we coordinate everything \cite{Graf2026382}. We have to make sure that the devices are working together smoothly without using up much bandwidth or computer power.

Hybrid and adaptive scheduling models need computer power because they change how things are set up on the fly. Models that use intelligence to decide how to use resources show that making smart scheduling decisions can reduce delays but they also need more processing power \cite{Mumtaz202596198}. So when we look at MAC protocols we have to think about more than just how stable the delaysre. We also have to think about whether they're feasible to implement if they can be scaled up and what they cost in terms of energy.

We are looking at the MAC design approaches and comparing them in Table~\ref{tab:comparison}. This table shows us the differences in how long things take how they work with a lot of people and how hard they are to set up.

\begin{table}[htbp]
\centering
\caption{Comparative analysis of low-latency MAC approaches for real-time wireless control systems synthesized from recent networking studies \cite{Sharma2023106898, Kalita20249365}.}
\label{tab:comparison}
\renewcommand{\arraystretch}{1.4}

\begin{tabular}{|p{3cm}|p{3.5cm}|p{3.5cm}|p{3.5cm}|}
\hline
Aspect & Deterministic MAC & Hybrid MAC & AI-Assisted MAC \\
\hline
Delay Guarantee & Strict bounded latency with predefined slots & Moderate bounds with adaptive reallocation & Adaptive latency optimization with learning-based control \\
\hline
Jitter Performance & Very low due to fixed scheduling & Controlled, variable under load & Reduced on average but may fluctuate during learning phase \\
\hline
Scalability & Moderate, depends on coordination overhead & High adaptability to dynamic traffic & High scalability with intelligent resource adjustment \\
\hline
Implementation Complexity & Medium, requires synchronization and slot management & Medium to high depending on configuration logic & High due to computational and training requirements \\
\hline
Industrial Suitability & Ideal for strict real-time control loops & Suitable for mixed critical and non-critical traffic & Promising for adaptive industrial environments \\
\hline
\end{tabular}
\end{table}


\section{Practical Insights and Use Cases}

\subsection{Industrial Automation Systems}

Industrial automation is a tough area for special protocols that help machines talk to each other quickly. In factories many machines need to communicate all the time to make sure everything is working correctly and safely. This means that sensors, controllers and actuators need to keep talking to each other.

Some special scheduling methods used in internet networks have shown that they can really help make sure that messages are delivered on time \cite{Kalita20249365}. This is important because it means that the machines can work together smoothly.

Industrial automation, like this uses something called medium access. This ensures that important commands are delivered quickly and on time which reduces the chance that something will go wrong and the production line will stop. Industrial automation needs this to work properly.

Network management is really important for industrial setups. Studies on management frameworks for 6TiSCH networks show that grouping devices and using a coordinator can make performance more consistent in environments where timing is crucial \cite{Graf2026382}. When you combine scheduling that always follows a set pattern with management strategies it helps systems grow and still keeps delays under control. These real world examples prove that the design of the MAC layer has an impact on how reliable industrial control systems are. The 6TiSCH networks and their management are critical, to this reliability.

In advanced factories where machines do most of the work things need to happen very fast. We are talking about things happening in milliseconds or even faster. If there is any delay in communication it can cause problems with the way the machines work together.

The people who make these factories work are looking at ways to make sure everything happens at the time. They are using systems that can connect wireless devices to other devices that need to work together perfectly \cite{Sasiain2025259}. This helps make sure that all the machines are working together smoothly even if they are connected in ways.

Some factories have already tried this. It works well. They use a system where each machine has a time slot to do its job and there is a central computer that makes sure everything runs smoothly. This makes the whole factory more reliable, which is important because everything needs to happen at the time.

So it is clear that we need to make sure the machines can talk to each other in a way that's fast and reliable especially in the factories of the future. We need to make sure that the machines can work together perfectly and that is why MAC-level determinism is so important, in factories.

\subsection{Robotics and Autonomous Systems}

Robotic systems need to talk to each other. They have parts that sense things and parts that control things. If there is a delay in this talk it can cause problems. The robot might not move right or follow the right path. It might even have trouble avoiding things in its way.

There are methods that help robots talk to each other faster. These methods are based on something called reinforcement learning. They work well even when things are changing quickly. One of these methods is called uplink scheduling. It helps reduce delays when robots are talking to each other \cite{Robaglia20241142}. This is really important, for robots that are moving around and working together.

Hybrid communication models are really helpful for robots to work together in time. They do this by combining a way of allocating time slots with flexible scheduling phases. Some studies have looked at how wireless networking works in areas and they found that controlling access to the network helps devices that are in different places work together better. For example a study, by Rico-Menendez shows this \cite{Rico-Menendez202554}. These methods make it possible for multiple robots to work together reliably which is important because they need to be able to send and receive both updates and updates that happen because something specific occurred. Hybrid communication models make this possible by allowing both types of updates to coexist.

\subsection{Smart Grids and Critical Infrastructure}

Smart grid systems need to communicate so they can monitor things find faults and manage energy that is spread out. This is very important for systems that use wireless technology. They need to be able to guarantee that messages are delivered on time in situations where it is really important that everything works correctly \cite{Sharma2023106898}.

The system that controls how messages are sent needs to be reliable so that important signals, for controlling the grid are sent on time which helps to make sure that energy is distributed in a way. Smart grid systems rely on this to work properly.

Industrial automation systems that use mobile communication technologies also get benefits from resource coordination. This is something that has been studied by Zeydan and others \cite{Zeydan20256808}.

Critical infrastructure networks usually have to deal with kinds of traffic. These networks need to have scheduling for control signals and flexible allocation for monitoring data.

When we look at world deployments we can see that well-designed MAC protocols make these systems more reliable and reduce the risk of something going wrong in time-sensitive infrastructure systems, like industrial automation systems.

\subsection{Automotive and Vehicular Time-Sensitive Networks}

Wireless communication that is completely predictable is becoming more important in cars and other vehicles. This is because they need to be able to send and receive information in time to stay safe. Things like systems that help drivers and cars that can talk to each other need to be able to get information quickly and reliably so they can make good decisions.

When cars are moving around and things are changing special methods can help make sure that important information gets through on time. For example setting aside times for certain messages can really reduce delays \cite{Al-Khatib2024139649}. This is really important when different parts of the system need to work and send information back and forth at the same time. Wireless communication that is completely predictable is crucial for things, like this to work properly. Advanced driver assistance systems and cooperative vehicular networks rely on this kind of communication to work safely and effectively.

The integration of wireless time-sensitive networking with vehicular systems also introduces mobility-related challenges. Maintaining deterministic performance while nodes frequently join or leave the network requires adaptive scheduling and synchronization management. Research on mobility support in wireless TSN environments highlights the importance of seamless handover mechanisms that preserve timing guarantees \cite{Avila-Campos202430687}. As industrial automation expands into autonomous transportation systems, low-latency MAC protocols will play a crucial role in ensuring operational safety and reliability.


\section{Challenges and Open Issues}

\subsection{Synchronization and Timing Precision}

Time synchronization is really important for computer networks. These networks need all the computers to have the time so they can talk to each other without interfering. This is especially true for networks that use a schedule to send information. If all the computers do not have the time they might send information at the same time and cause problems.

Researchers are looking at how to make sure all the computers have the time in big wireless networks. They think it is very hard to get all the computers to agree on the time even if it is just a tiny fraction of a second off \cite{Chandramouli202335150}. There are a reasons why this is hard. Sometimes computers get a little slower or faster over time which can throw off their clock. It also takes some time for information to travel between computers, which can cause problems. Different computers might have small differences in how they work, which can also cause timing problems. Time synchronization is crucial, for MAC protocols and achieving precise time synchronization is essential for these networks to function correctly.

We also need to think about how integration with communication systems that are very reliable can affect how things work together. Some research on communication, in new wireless systems shows that we need better timing to keep very low latency going \cite{Sharma2023106898}. If we want to make sure everything is synchronized without using much extra resources that is still something we do not know how to do well in wireless control systems that need to work in real time.

\subsection{Scalability and Network Density}

Scalability is a problem when industrial wireless networks get really big and complicated. When we assign time slots in an order it works well for networks that are not too large. When we have a lot of devices in a small area it becomes hard to coordinate everything.

In 6TiSCH networks letting devices assign their time slots can actually help the network grow without getting too slow. This is shown in the work by Kalita \cite{Kalita20249365}. However when we have a lot of devices we need to make sure they can all communicate without delays. This requires good coordination strategies, for the industrial wireless networks. Industrial wireless networks need strategies to manage all the devices.

Network management is really important for scalability. Some studies on managing time-sensitive networks using clustering show that working together in a structured way makes the system more robust. For example research like the one done by Graf shows this \cite{Graf2026382}. With these new ideas it is still hard to keep delays predictable when the network is changing and there are more devices. Network management frameworks like these are still a challenge for researchers to figure out. Network management is the key, to making this work.

\subsection{Energy Efficiency and Coexistence}

Wireless sensor networks have a problem with energy consumption. This is especially true for networks that run on batteries or get energy from their surroundings. The usual way of scheduling things often needs to be checking all the time and keeping an eye on the channels, which can use a lot of power. Some new ways of scheduling have been made for sensor networks that take into account how much energy is being used. These new ways show that we can use energy and still get the information we need on time by balancing when devices can send information and how much power they use \cite{vanLeemput2024180034}. Making protocols, for wireless sensor networks that work well and use energy is still a very hard problem to solve. Wireless sensor networks need to be designed in a way that they can send information quickly and use energy at the same time.

Living with wireless technologies makes it even harder for things to communicate in real time. In factories and other industrial places they often use different wireless systems that work on the same frequencies, which increases the chance of them interfering with each other. When we design systems for communication we need to make sure they can handle interference and work well together. This is really important, for making sure everything works reliably as we can see in the work of Zeydan and others \cite{Zeydan20256808}. Making sure that wireless communication works in a way even when there are a lot of other signals around is something that still needs to be figured out.

\subsection{Security and Resilience in Deterministic Wireless Control}

Security is a deal when we are designing low-latency MAC protocols. We have to think about this because real-time wireless control systems are used in important industries. If someone gets into the system without permission or interferes with it on purpose it can be very dangerous.

To keep these systems safe, there is a need to add some features like encryption and authentication at the MAC layer. This can make the system slower and it takes longer to send information.

Some people have done research on how to make sure communication happens in a way. They found out that we have to be very careful when we add security features so that we do not make the system too slow \cite{Sharma2023106898}. It is hard to make sure the system is both: secure and fast at the time. We have to balance security with speed. Security considerations like these are very important, in low-latency MAC protocol design.

Resilience against interference and network faults is equally important in industrial settings. Wireless networks deployed in production facilities may experience signal obstruction, multipath fading, or intentional jamming. Structured scheduling combined with adaptive channel allocation improves reliability under adverse conditions \cite{Kalita20249365}. However, maintaining deterministic guarantees during fault recovery or topology changes requires sophisticated coordination mechanisms. Future MAC designs must incorporate fault-tolerant scheduling strategies to sustain stable real-time control performance.


\section{Future Directions}

\subsection{Integration with 6G and Advanced Wireless Systems}

The move to 6G and better wireless systems is going to have an impact on how low-latency MAC protocols change over time. New ways of communicating that are very reliable and have controlled delays are being developed for services that need them \cite{Sharma2023106898}. If we combine these communication methods with MAC scheduling and future cellular systems, industrial control systems will be able to use big wireless networks and still keep strict timing rules. This is important for mission-critical services. The 6G technology and advanced wireless systems will make it possible for industrial control systems to work well with large-scale coverage.

Research on 5G technologies and time-sensitive networking shows that we can get deterministic performance in wireless environments if we reserve resources and schedule things carefully \cite{Sasiain2025259}. So future wireless control systems will probably use programmable radio access networks and structured allocation strategies to make sure delays are predictable. This means 5G technologies will work well with time-sensitive networking to give us better results.

As we get closer to having 6G wireless networks we need to make sure that our connections are really precise and that we can control how radios work. This is important because future networks will have to deal with strict rules about how long it takes for things to happen \cite{Sharma2023106898}. Some researchers propose deterministic communication paradigms that divide resources carefully and schedule transmissions so that they happen at the right time. If we can make our MAC protocols do this then we can use wireless networks for things like factories and still make sure that everything happens in real time. So over the next ten years we will probably see some big changes in how we design MAC protocols for deterministic networks. The way that cellular networks and industrial communication systems work will likely become more similar and this will change how we think about designing MAC protocols that can guarantee real-time performance.

\subsection{Artificial Intelligence in MAC Scheduling}

People are looking into ways that artificial intelligence can help make MAC-layer scheduling work. They are using intelligence to help decide how to share radio resources in next-generation wireless networks \cite{Mumtaz202596198}. This shows that artificial intelligence can make decisions that change based on the situation, which can help reduce latency when the traffic on the network is changing a lot. Artificial intelligence can help MAC protocols adjust the settings for sending data while still meeting the required timing. This is done by using intelligence to find the best settings so the MAC-layer scheduling can work well.

Reinforcement learning-based scheduling methods show that intelligent adaptation can really work well in situations where communication needs to be very reliable and not slow \cite{Robaglia20241142}. These methods let systems get ready for congestion and changes in traffic before they happen. But when we use intelligence in real-time control we have to be careful about how much work the computer has to do and how long it takes to make decisions so we do not disturb the tight timing that is needed for reinforcement learning-based scheduling methods to work properly in ultra-reliable low-latency communication scenarios.

\subsection{Cross-Layer and Deterministic Network Integration}

Future research directions also include working closely between control algorithms and communication protocols. Deterministic networking approaches across multiple domains show how important it is to carefully plan how resources are used so that things can be done on time \cite{Rico-Menendez202554}. If we can include scheduling policies that consider control requirements into MAC protocols then we can make the system more stable when it is running.

The integration of MAC protocols with distributed industrial automation architectures is another important area. Cloud-fog automation frameworks show how communication across hierarchical layers can improve responsiveness and reliability \cite{Jin20252917}. Wireless MAC protocols and distributed industrial automation architectures can work together effectively in these frameworks because coordinated communication helps the system respond faster and more reliably.

Continued research in cross-layer design and deterministic resource coordination will help develop wireless control systems that are scalable, robust, and capable of operating in real time. So wireless MAC protocols and distributed industrial automation architectures will be important for real-time wireless control deployments.

\subsection{Deterministic MAC in Integrated Sensing and Communication}

Future wireless systems will probably combine communication and sensing together as one system. This means that the same radio signals will be used for both data exchange and environmental awareness. In factories and robotic systems this can help maintain situational awareness while preserving real-time control performance. Emerging deterministic communication frameworks emphasize that bounded latency must be maintained even when sensing data shares transmission resources \cite{Sharma2023106898}.

The coexistence of sensing and control traffic introduces new scheduling challenges at the MAC layer. Resource allocation must ensure that sensing updates do not interfere with time-critical control packets. AI-assisted resource management approaches indicate that adaptive scheduling can dynamically prioritize latency-sensitive traffic while accommodating additional sensing flows \cite{Mumtaz202596198}. Developing MAC protocols capable of supporting both deterministic communication and integrated sensing will be a key research direction in future industrial wireless networks.


	\section{Conclusion}

Real-time wireless control systems need a way to communicate that is predictable and does not take too long. We discussed some protocols that can help with this. These protocols are designed for places like factories and other systems that rely on computers and machines. They have to work on time every time. Some scheduling mechanisms can make sure that things happen exactly when they are supposed to, which is very important for applications that cannot be late. Research in this area shows that when we carefully plan how resources are used it makes systems in factories and other places work reliably and consistently \cite{Rico-Menendez202554}. Real-time wireless control systems like these are very important for industries.

Comparative analysis further indicates that adaptive and hybrid MAC approaches can improve scalability while preserving acceptable delay bounds under varying traffic conditions. Emerging developments in deterministic communication for next-generation wireless systems suggest that integrating advanced timing coordination and programmable resource allocation will shape the future of industrial wireless control \cite{Sharma2023106898}. As industrial automation continues to transition toward fully wireless operation, advancing scalable and energy-efficient MAC protocol design will remain essential for achieving ultra-reliable low-latency performance.

    \bibliographystyle{plain}
    
	\bibliography{references}
	
\end{document}
